\documentclass[12pt]{article}

\usepackage{times}
\usepackage[T1]{fontenc}
\usepackage{hyperref}

\setlength{\parskip}{1em}\setlength{\parindent}{0pt}
\linespread{1.25}
\usepackage[margin=0.7in,top=0.5in]{geometry}\usepackage{fancyhdr}
\pagestyle{fancy}\lhead{\bf \name}\rhead{\bf \wustlkey}\cfoot{\thepage}

%%%%%%%%%%%%%%%%%%%%%%%%%%%%%%%%%%%%%%%%%%%%%%%%%%%%%%%%%%%%%%%%%%%


\usepackage{amsmath,graphicx}


\begin{document}
\title{Research Interests}
\date{\vspace{-10ex}}
\maketitle

Thank you for reading my research statement! I would love to share with you my personal interests in computer vision.


My personal interests can be divided into two parts. First one is 3D computer vision, especially 3D vision techniques in reconstruction, VR and AR. I am a crazy fan of video game and sci-fi movie, where plenty of real, high-quality 3D objects and land scenes are required. Building decent models not only cost a lot of money, which may not be affordable for some startup, but is also a very time-consuming job. Thus, rapid 3D reconstruction is of great vital for the development of entertainment industry. Besides, for VR/AR, motion capture and SLAM are also useful for game development. VR/AR techniques with extraordinary effects and useful functions may exist only in sci-fi movies or some specific fields, like video games and biomedical industry, and there will still be a long way for the VR/AR techniques to obtain a widespread use. Thus, it will be an excellent job to research on these areas and bring those techniques into real life. Making contribution to the things you like is awesome, and I will be great happy if my further work can help the development of the entertainment industry and VR/AR.


Then the second one is image texture synthesis. Texture synthesis is one of the most creative job I have worked on. I was greatly surprised when I saw the amazing results of style transfer at the first time. There are so many areas where image texture synthesis can help, like style transferring, high resolution image reconstruction, image inpainting and extrapolation, etc. Besides, generative adversarial network brings plenty of unbelievable but also hard-to-control performance. I am really interested in how to control and optimize GAN to generate ``real'' images with high quality and high resolution. I like art, oil painting and photography, and it will be an extremely amazing experience to bring the powerful machine learning and image processing algorithms to these fields.


I have respectively made some demos and experiments on these two fields. I participated in research on 3D model recognition, re-identification, neural style tranferring and 3D reconstruction. More projects can be found in my github and \href{https://lewkesy.github.io}{personal home page}.


Thank you again for reading my research interests. Looking forwards to hearing from you.

\end{document}
